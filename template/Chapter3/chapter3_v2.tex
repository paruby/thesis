%!TEX root = ../thesis.tex
%*******************************************************************************
%****************************** Third Chapter **********************************
%*******************************************************************************
\chapter{Causal Abstractions}

% **************************** Define Graphics Path **************************
\ifpdf
    \graphicspath{{Chapter3/Figs/Raster/}{Chapter3/Figs/PDF/}{Chapter3/Figs/}}
\else
    \graphicspath{{Chapter3/Figs/Vector/}{Chapter3/Figs/}}
\fi

\emph{This chapter is largely based on the paper:}


\begin{quote}
\fullcite{rubenstein2017causal_fullcite}.
\end{quote}

%\citep{rubenstein2017causal}
%\cite{rubenstein2017causal}
%\citet{rubenstein2017causal}

\emph{Sections ... are an in-depth introduction to causality, motivating and explaining the setting considered by the paper.
Sections ... are mostly based on the paper. 
Section ... discusses how this work has influenced the research community and briefly discusses work that builds on it.}



\section{Introduction}

Much of machine learning concerns the statistical relationships between random variables. In this context, the word \emph{statistical} refers to the assumptions that a fixed but unknown probability distribution exists from which observed data are sampled \emph{independently and identically distributed (\iid)}, and that new data at `test time' will similarly be drawn \iid~from this distribution.
Classification is the canonical example of this, where given a set of \iid~samples from a joint distribution $\PP_{XY}$ over input $X$ and discrete target $Y$, the goal is to learn the conditional distribution $\PP_{Y|X}$ giving the probability distribution over targets for each possible input. Other problems such as density estimation can be phrased similarly.
%Given a set of \emph{i.i.d.} samples from a distribution $\PP_X$, the goal of \emph{density estimation} is to learn an approximation to $\PP_X$.

Despite the great empirical successes of machine learning in practical and applied settings in recent years, there remain problems of interest that cannot be cast directly in the framework described above. 
The framework is limited in that it presupposes the existence of a single joint distribution over all of the random variables of interest, with the operations of marginalisation and conditioning then providing the relationships connecting any subset of variables.
But there are many examples of problems for which a single joint distribution over all variables does not suffice. 
For many questions of scientific interest, this is because the problem either implicitly or explicitly concerns an \emph{intervention} or \emph{action} in the world that changes the joint distribution over the observable variables.

For example, we may be interested to understand the influence of diet on longevity, with the aim of improving public health by encouraging people to eat healthily. 
One might find the consumption of expensive imported fruits to be correlated with a longer life. This may well be due to the nutritiousness of such fruits; it could equally well be due to the fact that only wealthy people can afford such a diet, and that such wealth entails better access to medical treatment, sports facilities for exercise and so on. In the former case, intervening in the world by reducing tariffs on imported fruits to encourage their consumption would have a positive effect on public health; in the latter, not.
Similarly, we may observe in the population that taking over-the-counter painkillers is associated with an elevated risk of heart disease. This might be because such painkillers have a negative effect on the cardiovascular system, in which case acting to reduce access to such painkillers might have a positive impact on health outcomes. But if instead the association is because of that fact that people who already have poor health, and thus heightened risk of heart disease, tend to take more painkillers, then such a policy might have little effect other than to increase overall suffering.

These questions are concerned with understanding causal, not statistical, relationships in the world. 
The aim of \emph{causality} is to study causal influence through the lens of a formal mathematical language, in much the same way that statistical machine learning uses the language of probability. 
\emph{Causal inference} or \emph{causal discovery}, a large part of the causality literature, concerns the identification of causal relationships using data.
As the examples above demonstrate, this can be highly non-trivial, with the well-known phrase ``correlation does not imply causation'' standing testament to the simultaneous difficulty and ubiquity of this problem.
Since correlation (or statistical dependence more generally) is a symmetric relation, an asymmetric causal relationship can never be inferred without other prior knowledge. 
Moreover, simply identifying that two quantities tend to co-occur does not itself imply a causal relation between the two, since both could be causally influenced by a third.

While causal inference is an important problem with a wide variety of applications ranging from astronomy to neuroscience and economics [**citations**], the main contribution of this chapter is to extend and provide greater understanding of \emph{Structural Equation Models (SEMs)}, one of the popular mathematical frameworks for formalising causal relationships between random variables and interventions, along with the variety of probability distributions these entail. 
In particular, this work seeks to understand the implications of modelling causal structure at a different level of \emph{abstraction} compared to the `truth'. For instance, causal influence between variables of interest may be mediated by irrelevant variables that are ignored; interactions between low-level variables may instead be modelled at a macroscopic level, similar to how temperature and pressure arise as macroscopic properties of a large number of gaseous particles; and though time invariably plays a role in any causal influence in the real world, mathematical models of causal structure may often omit explicit reference to it.
This work additionally sheds new light on \emph{cyclic} SEMs, a previously poorly understood family of causal models, and has has implications to existing causal inference algorithms by providing a framework for understanding the limitations of previous approaches.


The next sections provide a comprehensive background sufficient to understand the novel contribution of this chapter.
In Section ??, we will introduce SEMs, followed in Section ?? by an overview of approaches to causal inference using SEMs. 
In Section ?? we will discuss the implicit assumptions in these approaches to causal inference, which will lead us to discussing causal abstractions in Sections ?? to ??.
Section ?? discusses the influence that this work has had on the research community in the $\sim 2$ years since its publication as well as recent work by others that directly builds on it.

%Best summarised by the well known phrase ``correlation does not imply causation'', while causal influence between two random variables typically implies statistical dependence, the reverse is not true.
%At its heart, causality and statistical machine learning differ in that the former concerns itself with interventions in the world. 





%\begin{itemize}
%\item There are interesting problems that don't fall into above framework that are nonetheless about learning from data.
%\item For instance scientific questions that are causal in nature, which scientists investigate by gathering and analysing data. These often are to do with decision making. For instance, given a patient with some condition, should we give them drug A or drug B? Here we are interested in the causal influence of the choice of drug on health outcome, not the statistical correlations. The reasons for this are best illustrated with an example.
%\item Suppose that drug A is a well established treatment for a condition, while drug B is new and experimental. It may be that only those patients that are very  
%\end{itemize}


\section{Structural Equation Models: A Language for Causality}

\emph{This section introduces Structural Equations Models (SEMs), a mathematical formalism used to model causal influence. Later in the chapter an extension to the classical SEM will be presented; as such, in this section we present classical SEMs in a slightly different way compared to typical treatments.}
\\

\noindent An SEM over a tuple of random variables $X = (X_1, \ldots, X_N)$ consists of equations so that each $X_i$ is written as a function of a subset of the other $X_j$ and an \emph{exogenous noise variable} $E_i$. More formally:

\begin{definition}[Structural Equation Model (SEM)]
Let $X = (X_1, \ldots, X_N)$ and $E = (E_1, \ldots E_N)$ with each $X_i$, $E_i$ taking value in $\R$. An SEM $\mathcal{M}_X$ over $X$ is a tuple $(\mathcal{S}_X, \mathbb{P}_E)$ where
\begin{itemize}
	\item $\mathcal{S}_X$ is a set of structural equations of the form $X_i = f_i(X_{\pa(i)}, E_i)$ for $i=1,\ldots,N$, where the functions $f_i$ are measurable, $\pa(i) \subset \{1,\ldots,N\}$ and $X_{\pa(i)}$ is the corresponding subset of the variables $X$.
	\item The variables $E = (E_1,\ldots,E_N)$ have distribution $\mathbb{P}_E$ which factorises, i.e. the $E_i$ are independent.
	\item The \emph{causal graph} $\mathcal{G}$, the directed graph with nodes $X_i$ and edges $X_i \to X_j$ if and only if $i \in \pa(j)$, is acyclic.
\end{itemize}
\end{definition}

\paragraph{Remark:}
The requirement that $\mathcal{G}$ be acyclic ensures that the SEM implies a well defined distribution $\mathbb{P}_X$ over the variables $X$. This is known as the \emph{observational distribution}. We will discuss this fact further next. Later, we will discuss the additional fact that, in combination with the requirement that the noise variables $E_i$ are independent, $\mathcal{G}$ being acyclic implies a correspondence between the conditional independence properties of the joint distribution $\mathbb{P}_X$ and $\mathcal{G}$, which is useful for learning $\mathcal{G}$ from data (i.e. causal inference). 

\begin{lemma}[Well defined observational distribution]
An SEM implies a well-defined observational distribution $\mathbb{P}_X$ over $X$.
\end{lemma}
\begin{proof}
For any particular value $e$ of the noise variables $E$, there is a unique vector $x(e)$ so that $(x(e), e)$ solves the structural equations $\mathcal{S}_X$. To see this, observe that acyclicity of $\mathcal{G}$ means that the structural equations can be solved recursively, beginning with variables with no parents. It follows that each $x_i$ can be written as a function of $e_{\text{anc}(i)}$, where $\text{anc}(i)$ are the indices of the \emph{ancestors} of $X_i$ in $\mathcal{G}$, that is the $X_j$ for which there exists a path $X_j \to X_i$....

continue this proof. Use the fact that $f_i$ are measurable.
\end{proof}

As discussed in the introduction, an important part of causal relationships is a notion of behaviour under \emph{interventions}. SEMs are equipped with a formal notion of such interventions which we call \emph{perfect interventions}. The idea is that intervening on a variable makes a change to the function determining its value in the observational setting.  Although there may be an effect on the \emph{distributions} over variables downstream of the intervened variable, the functions determining those variables are unchanged. Such an intervention is realised simply by replacing the structural equation of the intervened variable. This notion is easily generalised to interventions on multiple variables by replacing all of the corresponding equations. Formally:

\begin{definition}[Perfect interventions and the do-operator]
	Let $\mathcal{M}_X$ be a SEM, $i \in \{1,\ldots,N \}$ and $x_i \in \mathbb{R}$. The perfect intervention setting $X_i$ to take value $x_i$ is denoted $\doop(X_i = x_i)$ and is implemented by replacing the $i$th equation with $X_i = x_i$. The resulting set of structural equations is denoted $\mathcal{S}_X^{\doop(X_i=x_i)}$ and the resulting SEM $\mathcal{M}_X^{\doop(X_i=x_i)} = (\mathcal{S}_X^{\doop(X_i=x_i)}, \mathbb{P}_E)$.
	Perfect interventions over two or more variables are also valid, which are denoted for instance as $\doop(X_i = x_i, X_k = x_k)$.
\end{definition}

\paragraph{Remark:} We note in passing that the notion of a perfect intervention can also be extended to a notion of \emph{imperfect} or \emph{stochastic intervention} in which the intervened variable is set equal to some random variable rather than a constant. Formally, this is no different from the perfect case other than needing to additionally introduce the new random variables. We will not discuss such interventions further.

\paragraph{Remark:} An intervened SEM is still just a SEM, since it has structural equations and a distribution over exogenous variables. The only difference is that the functions corresponding to an intervened variable $X_i = f_i(X_{\pa(i)}, E_i)$ will have $\pa(i) = \emptyset$ and $f_i$ will be a constant function. Moreover, the causal graph $\mathcal{G}_{\doop(\ldots)}$ has the same nodes but a \emph{subset} of edges compared to $\mathcal{G}$, and thus inherits acyclicity. This implies the following lemma.

\begin{lemma}[Well defined interventional distribution for any perfect intervention]
	Any perfect intervention on a SEM (i.e. any subset of variables set to any particular values) implies a well-defined interventional distribution $\mathbb{P}^{\doop(...)}_X$ over $X$.
\end{lemma}
\begin{proof}
	As discussed in the remark above, $\mathcal{M}_X^{\doop(...)}$ is a valid SEM. Thus, Lemma ?? applies to $\mathcal{M}_X^{\doop(...)}$.
\end{proof}
	
\paragraph{Remark:} SEMs can thus be thought of as a way to model not just a single distribution over the variables of interest, but an entire family of related distributions, one for each possible perfect intervention.


\paragraph{Example:} Give example of simple SEM, one or two interventions and the distributions implied by all of them.

\subsection{Connections between SEMs and Bayesian networks}

SEMs as described in the previous section are essentially equivalent to Bayesian networks, a class of graphical models.
\begin{definition}
A Bayesian network over variables $X = (X_1,\ldots, X_N)$ with directed acyclic graph $\mathcal{G}$ specifies a joint distribution over $X$ as a product of simple conditional distributions
\begin{align*}
	p(X_1,\ldots,X_N) = \prod_{i=1}^N p(X_i | X_{\pa(i)})
\end{align*}
where the nodes of $\mathcal{G}$ correspond to the variables $X$ and there is an edge $X_j \to X_i$ in $\mathcal{G}$ if and only if $j \in \pa(i)$.

Bayesian networks can be endowed with a similar notion of perfect intervention as SEMs. To model the effect of the perfect intervention $\doop(X_i = x_i)$, in the resulting joint distribution the factor $p(X_i | X_{\pa(i)})$ is replaced with a dirac delta distribution $\delta_{X_i = x_i}$.
\end{definition}

\paragraph{Remark (equivalence of SEMs and Bayesian networks):} Any SEM induces a Bayesian network with the same graph $\mathcal{G}$.
To see this, observe that for any fixed value of $X_{\pa(i)}$, the equation $X_i = f(X_{\pa(i)} , E_i)$ in combination with the distribution over $E_i$ induces a distribution over $X_i$ which corresponds to $p(X_i | X_{\pa(i)})$. The observational distributions of the SEM and Bayesian network are then equal. Further, since the equation $X_i=x_i$ associated with the intervention $\doop(X_i = x_i)$ corresponds to the distribution $\delta_{X_i = x_i}$, and similar for interventions or arbitrary subsets of variables, the interventional distributions also agree.

Showing the reverse, namely that any Bayesian network induces a SEM, is somewhat more complicated. In the case that all variables are discrete, this can be shown relatively straightforwardly \cite{druzdzel1993causality}. The intuition is that for any fixed value for $x_{\pa(i)}$,  the distribution $p(X_i | X_{\pa(i)}=x_{\pa(i)})$ can be written as a transformation $g_{x_{\pa(i)}}(E_i)$ of a uniformly distributed random variable $E_i$. We can then define the function $f_i$ of the constructed SEM as $f_i(x_{\pa(i)}, e_i) = g_{x_{\pa(i)}}(e_i)$, which is straightforwardly measurable as all spaces are discrete. [**Check this**]

In the general case, however, one quickly runs into very technical measure-theoretical arguments due to the requirement that the functions $f_i$ be measurable. [Basically, the issue is that you need to ensure that $E_i$ can be transformed into $p(X_i | X_{\pa(i)}=x_{\pa(i)})$ for all choices of $x_{\pa(i)}$. You get a different transformation$g_{x_{\pa(i)}}$ for each value of $x_{\pa(i)}$. Then you need to show that defining $f_i(x_{\pa(i)}, e_i) = g_{x_{\pa(i)}}(e_i)$ results in $f_i$ being measurable. If $f_i$ is continuous in each of its arguments then paper cited in second answer of \url{https://math.stackexchange.com/questions/215215/showing-a-function-of-two-variables-is-measurable} says that it is measurable. But in the general case I'm not sure what to do. 

I think it suffices to say that in most typical use cases, the conditional probability distributions will be specified in some way that makes them amenable to directly rewriting as an SEM. For example, if conditional distribution is Gaussian with mean and covariance dependent on parents, the reparameterisation trick is precisely writing this as a SEM.]
\\

\noindent
%Since the connection between SEMs and Bayesian networks is so close, it is instructive to first discuss their two main differences.
There are two main differences between SEMs and Bayesian networks.
The first is philosophical: SEMs are usually explicitly used with the modelling of causal structure in mind, while Bayesian networks are often not. For instance, variables in a Bayesian network may be latent variables that have no direct physical meaning.
The second is that in an SEM, the noise variables $E$ are explicitly modelled, while in a Bayesian network they are not. Exceptions to this have arisen in recent years with the use of the \emph{reparameterisation trick} as a way to train Bayesian networks via stochastic gradient methods \citep{kingma, rezende}, though as the name suggests this is generally viewed as a computational `trick' rather than a general strategy of mathematical modelling.
The explicit use of a noise variable enables \emph{counterfactual reasoning}. That is, once the variables have been observed, the noise values for the variables can generally be inferred. This means that one can answer questions such as ``what would have happened had the intervention $\doop(X_i=x_i)$ been performed?'', as illustrated by the following example:

\paragraph{Example:} todo.

Note that different SEMs may imply the same set of observational and interventional distributions, but nonetheless be counterfactually non-equivalent. That is, they may produce different answers to the same counterfactual question, despite implying the same observational and interventional distributions. This means that such models cannot be distinguished based on data, be it in the observational or interventional setting. Therefore one cannot hope to be able to answer counterfactual questions based on data without prior knowledge to distinguish between counterfactually non-equivalent models.



\section{Methods of Causal Inference}
\section{What are causal variables?}
\section{Transformations between Structural Equation Models}
\section{Examples of transformations}
\section{Discussion and work building on this}










































