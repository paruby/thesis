%!TEX root = ../thesis.tex
%*******************************************************************************
%****************************** Third Chapter **********************************
%*******************************************************************************
\chapter{Causal Abstractions}

% **************************** Define Graphics Path **************************
\ifpdf
    \graphicspath{{Chapter3/Figs/Raster/}{Chapter3/Figs/PDF/}{Chapter3/Figs/}}
\else
    \graphicspath{{Chapter3/Figs/Vector/}{Chapter3/Figs/}}
\fi

High level explanation of chapter and outline of contents.

This chapter is based on the paper \emph{Causal Consistency of Structural Equation Models} published at UAI 2017.



\section{Introduction to Causality}

Much of machine learning concerns the statistical relationships between random variables. In this context, the word \emph{statistical} refers to the assumptions that a fixed probability distribution exists, that observed data are independent and identically distributed (\iid) samples from this distribution, and that new data at `test time' will similarly be drawn \iid~from the same distribution.
Classification is the canonical example of this, where given a set of \iid~samples from a joint distribution $\PP_{XY}$ over input $X$ and discrete target $Y$, the goal is to learn the conditional distribution $\PP_{Y|X}$ giving the probability distribution over targets for each possible input. Other problems such as density estimation can be phrased similarly.
%Given a set of \emph{i.i.d.} samples from a distribution $\PP_X$, the goal of \emph{density estimation} is to learn an approximation to $\PP_X$.

Despite the great empirical success of machine learning in practical and applied settings in recent years, there remain problems of interest that cannot be cast directly in the framework described above. 
Consider the following 


\section{Structural Equation Models: A Language for Causality}
\section{Methods of Causal Inference}
\section{What are causal variables?}
\section{Transformations between Structural Equation Models}
\section{Examples of transformations}
\section{Discussion and work building on this}
