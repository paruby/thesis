%!TEX root = ../thesis.tex
% ******************************* Thesis Appendix B ********************************

\chapter{Additional Materials for Chapter \ref{chapter:ica}}\label{chapter:appendix-ica}

\section{Proofs for one noiseless view results (Section \ref{sec:onenoisless})}
%\section{Proof of Theorem \ref{thm:noiseless1} and Corollary \ref{crl:noiseless1}}
\label{appendix:thm_noiseless}

\subsection{Proof of Theorem \ref{thm:noiseless1}}\label{appendix:proof-thm1}

\begin{reptheorem}{thm:noiseless1}
	The difference between the log joint probability and log product of marginals of the observed variables in the model given in Equations \ref{eq:indep}-\ref{eq:sem2_1} admits the following factorisation:
	\begin{align}
	&\log p({x}_1, {x}_2) - \log p({x}_1) p({x}_2) \nonumber \\
	&= \log p({x}_2 | {x}_1) - \log p({x}_2) \nonumber\\
	&= \left(\sum_i \alpha_i(z_{i}, g_i(z_i, n_i)) + \log \det J \right) \nonumber\\
	&\qquad - \left( \sum_i \delta_i(g_i(z_i, n_i)) + \log \det J\right) \nonumber\\
	&= \sum_i \alpha_i(z_{i}, g_i(z_i, n_i)) - \sum_i \delta_i(g_i(z_i, n_i)) \,
	\end{align}
	where $z_i=f^{-1}_{1i}({x}_1)$, $g_i=f^{-1}_{2i}({x}_2)$,
	and $J$ is the Jacobian of the transformation $f^{-1}_2$ (note that the introduced Jacobians cancel).
	Suppose that
	\begin{enumerate}
		\item $\alpha$ satisfies the \emph{Sufficiently Distinct Views} assumption.
		\item A classifier is trained to discriminate between
		\begin{align*}
		(X_{1},X_{2}) \text{ vs. } (X_{1},X_{2}^{*})\,,
		\end{align*}
		where $({X}_{1},{X}_{2})$ correspond to the same realisation of $Z$ and $({X}_{1},{X}_{2}^{*})$ correspond to different realisations of ${Z}$.
		\item The classifier minimises the logistic regression loss, and is constrained to use a regression function of the form
		\begin{equation*}
		r({x}_{1},{x}_{2})=\sum_{i}\psi_{i}(h_{i}({x}_{1}),{x}_{2})
		\end{equation*}
		where ${h} =(h_{1}, \ldots, h_{n})$  is invertible, smooth and has smooth inverse.
	\end{enumerate}
	
	Then, in the limit of infinite data and with universal approximation capacity, $h$ inverts ${f}_1$ in the sense that the $h_{i}(X_1)$ recover the independent components of $Z$ up to component-wise invertible transformations.
\end{reptheorem}

This proof is inspired by the techniques employed by \cite{hyvarinen19a}.

\begin{proof}

The goal is to show that for the optimal classifier, $h_i(X_1)$ depends on exactly one component of $Z$. 

%	We have to
%	show that, upon convergence, $h_{i}(\bm{x}_{1})$ are s.t.
%	\begin{align*}
%	h_{i}(\bm{x}_{1})\independent h_{j}(\bm{x}_{1}),\forall i\neq j
%	\end{align*}
	
We begin by writing the difference in log-densities of the two classes
	\begin{align*}
	\sum_{i}\psi_{i}(h_{i}(x_{1}),x_{2})  &=\sum_{i}\alpha_{i}({f}_{1,i}^{-1}({x}_{1}), {f}_{2,i}^{-1}({x}_{2})) -\sum_{i}\delta_{i}( {f}_{2,i}^{-1}({x}_{2})).
	\end{align*}
Making the change of variables
	\begin{align*}
	{y} & ={h}({x}_{1}),\\
	{v}({y}) & ={f}_{1}^{-1}({h}^{-1}({y})),\\
	{t} & = {f}_{2}^{-1}({x}_{2})),
	\end{align*}
means that the first equation can be rewritten in the following form:
	\begin{align}
	\sum_{i}\psi_{i}(y_{i},{x}_{2})=&\sum_{i}\alpha_{i}(v_{i}({y}), t_{i}) -\sum_{i}\delta_{i}( t_{i}). \label{eqn:ica-appendix:1}
	\end{align}
	
Take derivatives with respect to $y_j$, $y_{j'}$, $j \neq j'$,  of each side of this equation. 
Adopting the notation from the SDV assumption definition
	\begin{align*}
	\alpha'_{i}(y_i, t_i)&= \partial \alpha_{i}(y_i, t_i)/\partial y_i,\\
	\alpha''_{i}(y_i, t_i)&=\partial^2 \alpha_{i}(y_i, t_i)/\partial y_i^2, \\
	{w}_{\alpha}({y}, {t}) &= (\alpha''_{1}, \ldots, \alpha''_{D}, \alpha'_{1}, \ldots,\alpha'_{D}),
	\end{align*}
 and furthermore defining 
	\begin{align*}
	v^j_i({y})&=\partial v_i({y})/\partial y_j,\\
	v^{jj'}_i({y})&= \partial^2 v_i({y})/\partial y_j \partial y_{j'},
	\end{align*}
	we have
	\begin{align*}
	0 = \sum_{i} \alpha''_{i}(v_{i}({y}), t_{i})v^j_i({y})v^{j'}_i({y}) + \alpha'_{i}(v_{i}({y}),  t_{i})v^{jj'}({y}).
	\end{align*}
	The left-hand side of this equation is $0$ because each term $\psi_i$ in Equation \ref{eqn:ica-appendix:1} depends on exactly one $y_i$, and partial derivatives are taken with respect to two different $y_i$ and $y_j$.

	If we now rearrange our variables by defining vectors ${a}_i({y})$ collecting all entries $v_i^j({y})v_i^{j'}({y})$, $j=1, \ldots, n$, $j'=1, \ldots, j-1$, and vectors ${b}_i({y})$ with the variables $v_i^j({y})v_i^{j'}({y})$, $j=1, \ldots, n$, $j'=1, \ldots, j-1$, the above equality can be rewritten as
	\begin{align*}
	\sum_{i} \alpha''_{i}(v_{i}({y}), t_{i}){a}_i({y}) + \alpha'_{i}(v_{i}({y}),  t_{i})){b}_i({y})=0\,.
	\end{align*}
	
This can in turn be rewritten in matrix form,
	\[
	{M}({y}){w}({y}, {t})=0\,,
	\]
	where ${M}({y}) = ({a}_1({y}), \ldots,  {a}_D({y}), {b}_1({y}), \ldots, {b}_D({y})) $ and ${w}({y}, {t}) = (\alpha''_{1}, \ldots, \alpha''_{D}, \alpha'_{1}, \ldots,\alpha'_{D})$. ${M}({y})$ is therefore a $D(D-1)/2 \times 2D$ matrix, and ${w}({y}, {t})$ is a $2D$ dimensional vector.
	
	To show that ${M}({y})$ is equal to zero, we invoke the SDV assumption.
	This implies the existence of $2D$ linearly independent ${w}({y}, {t}_j)$.
	It follows that
	
	\[
	{M}({y})[{w}({y}, {t}_1), \ldots, {w}({y}, {t}_{2D})]=0\,,
	\]
	
	and hence ${M}({y})$ is zero by elementary linear algebraic results.
	It follows that $v_i^j({y})\not=0$ for at most one value of $j$, since otherwise the product of two non-zero terms would appear in one of the entries of ${M}({y})$, thus rendering it non-zero.
	Thus $v_i$ is a function only of one $y_j$.
	
	Observe that ${v}({y}) = {z}$.
	We have just proven that $v_i(y_{\pi(i)}) = z_i$ where $\pi$ is some permutation.
	Since $v_i$ is invertible, it follows that $h_{\pi(i)}({x}_{1}) = y_{\pi(i)} = v_i^{-1}(z_i)$ and hence the components of ${h}({x}_{1})$ recover the components of ${z}$ up to the invertible component-wise ambiguity given by ${v}$, and the permutation ambiguity.
	
\end{proof}

\subsection{Proof of Corollary \ref{crl:noiseless1}}\label{appendix:proof-cor2}

\begin{repcorollary}{crl:noiseless1}
	Consider the setting of Theorem \ref{thm:noiseless1} with the alternative factorisation of the log joint probability
	\begin{align}
	&\log p({x}_1, {x}_2) - \log p({x}_1) p({x}_2) \nonumber \\
	&= \log p({x}_1 | {x}_2) - \log p({x}_1)\nonumber \\
	&= \sum_i \gamma_i(z_i, g_i(z_i, n_i)) - \sum_i \beta_i(z_i)) \,.
	\end{align}
	Suppose that ${\gamma}$ satisfies the SDV assumption.
	Replacing the regression function with
	\begin{equation*}
	r({x}_{1},{x}_{2})=\sum_{i}\psi_{i}({x}_{1}, h_{i}({x}_{2}))
	\end{equation*}
	results in ${h}$ inverting ${f}_2$ in the sense that the $h_{i}({X}_2)$ recover the independent components of the ${g}({Z}, {N})$ up  to component-wise invertible transformations.
\end{repcorollary}

\begin{proof}
	This follows exactly by repeating the proof of Theorem \ref{thm:noiseless1} where the roles of $x_1$ and ${x}_2$ are exchanged and the regression function in the statement of the corollary is used.
\end{proof}


\section{Proofs for two noisy view results (Section \ref{sec:constrained})}

\subsection{Proof of Theorems \ref{thm:demixing} and \ref{thm:two-noisy-views}}
\label{appendix:thm1}

Theorem \ref{thm:demixing} is a special case of Theorem \ref{thm:two-noisy-views} by considering the case $\bm{g}_1(\bm{z}, \bm{n}_1) = \bm{z}$.
We therefore prove only the more general Theorem \ref{thm:two-noisy-views}.

\medskip

\begin{reptheorem}{thm:two-noisy-views}
	Suppose that ${\eta}$ and ${\lambda}$ satisfy the SDV assumption.
	The algorithm described in Theorem \ref{thm:noiseless1} with regression function specified in Equation \ref{eqn:double-regression-fn} results in ${h}_1$ and ${h}_2$ inverting ${f}_1$ and ${f}_2$ in the sense that the $h_{1,i}({X}_1)$ and $h_{2,i}({X}_2)$ recover the independent components of ${g}_1({Z}, {N}_1)$ and ${g}_2({Z}, {N}_2)$ up to two different component-wise invertible transformations. Furthermore, the two representations are aligned, i.e. for $i\not=j$,
	\begin{equation*}
	h_{1,i}({X}_{1})\independent h_{2,j}({X}_{2}).
	\end{equation*}
\end{reptheorem}


\begin{proof}
This proof is similar to that of Theorem \ref{thm:noiseless1}.

The goal is to show that for the optimal classifier, $h_{1i}(X_1)$ and $h_{2i}(X_2)$ each depend on exactly one component of $Z$. 
Moreover, in order to show that the representations are aligned, we will show that 
	\begin{align}
	h_{1,i}({x}_{1})\independent h_{2,j}({x}_{2}),\forall i\neq j. \label{eq:inds_twov_3}
	\end{align}
	
	We start by exploiting Equations \ref{eq:noisylogdens_1} and \ref{eq:noisylogdens_2} to write the difference in log-densities of the two classes
	\begin{align}
	\sum_{i}\psi_{i}(h_{1,i}({x}_{1}),h_{2,i}({x}_{2}))
	=&\sum_{i}\eta_{i}({f}_{1,i}^{-1}({x}_{1}), {f}_{2,i}^{-1}({x}_{2})) - \sum_{i}\theta_{i}({f}_{1,i}^{-1}({x}_{1})) \label{eq:first_factorization}\\
	=&\sum_{i}\lambda_{i}({f}_{2,i}^{-1}({x}_{2}), {f}_{1,i}^{-1}({x}_{1})) - \sum_{i}\mu_{i}({f}_{2,i}^{-1}({x}_{2}))\label{eq:2nd_factorization}
	\end{align}
	and make the change of variables
	\begin{align*}
	{y} & ={h}_1({x}_{1}),\\
	{t} & ={h}_2({x}_{2}),\\
	{v}({y}) & ={f}_{1}^{-1}({h}_1^{-1}({y})),\\
	{u}({t}) & ={f}_{2}^{-1}({h}_2^{-1}({t})).
	\end{align*}
	Equation \ref{eq:first_factorization} can thus be rewritten as
	\begin{align}
	\sum_{i}\psi_{i}(y_{i},t_{i}) 
	&=\sum_{i}\eta_{i}(v_i({y}), u_i({t}))
	-\sum_{i}\theta_{i}(v_i({y}))\label{eq:logistic}
	\end{align}

	To show that $h_{1i}(X_1)$ depends on exactly one component of $Z$, we will show that
	\begin{equation}
	\label{eq:v_onev}
	v_{i}({y})  = v_{i}(y_{\pi(i)})\\
	\end{equation}
	for some permutation of the indices $\pi$ with respect to the indexing of the sources ${z} = (z_1, \ldots, z_D)$.
	
	Taking derivatives with respect to $y_j$, $y_{j'}$, $j \neq j'$,  of equation \ref{eq:logistic} yields
	\begin{align*}
	0 = \sum_{i} \eta''_{i}(v_{i}({y}),u_{i}({t}))v^j_i({y})v^{j'}_i({y}) + \sum_{i}\eta'_{i}(v_{i}({y}), u_{i}({t}))v^{jj'}({y}).
	\end{align*}
	Define vectors ${a}_i({y})$ collecting all entries $v_i^j({y})v_i^{j'}({y})$, $j=1, \ldots, n$, $j'=1, \ldots, j-1$, and vectors ${b}_i({y})$ with the variables $v_i^j({y})v_i^{j'}({y})$, $j=1, \ldots, n$, $j'=1, \ldots, j-1$, the above equality can be rewritten as
	
	\begin{align*}
	0 = \sum_{i} \eta''_{i}(v_{i}({y}),u_{i}({t})){a}_i({y}) + \eta'_{i}(v_{i}({y}), u_{i}({t})){b}_i({y}).
	\end{align*}
	
	As in the proof of Theorem \ref{thm:noiseless1}, this can be rewritten in matrix form as
	
	\begin{equation}
	\label{eq:matrixmult}
	{M}({y}){w}({y}, {t})=0\,,
	\end{equation}
	
	where ${M}({y}) = ({a}_1({y}), \ldots,  {a}_D({y}), {b}_1({y}), \ldots, {b}_D({y})) $ and ${w}({y}, {t}) = (\eta''_{1}, \ldots, \eta''_{D}, \eta'_{1}, \ldots,\eta'_{D})$. ${M}({y})$ is therefore a $D(D-1)/2 \times 2D$ matrix, and ${w}({y}, {t})$ is a $2D$ dimensional vector.
	
	To show that ${M}({y})$ is equal to zero, we invoke the SDV assumption on ${\eta}$.
	This implies the existence of $2D$ linearly independent ${w}({y}, {t}_j)$.
	It follows that
	
	\[
	{M}({y})[{w}({y}, {t}_1), \ldots, {w}({y}, {t}_{2D})]=0\,,
	\]
	
	and hence ${M}({y})$ is zero by elementary linear algebraic results.
	It follows that $v_i^j({y})\not=0$ for at most one value of $j$, since otherwise the product of two non-zero terms would appear in one of the entries of ${M}({y})$, thus rendering it non-zero.
	Thus $v_i$ is a function only of one $y_j = y_{\pi(i)}$.
	
	Observe that ${v}({y}) = {z}$.
	We have just proven that $v_i(y_{\pi(i)}) = z_i$.
	Since $v_i$ is invertible, it follows that $h_{\pi(i)}({x}_{1}) = y_{\pi(i)} = v_i^{-1}(z_i)$ and hence the components of ${h}({x}_{1})$ recover the components of ${z}$ up to the invertible component-wise ambiguity given by ${v}$, and the permutation ambiguity.
	
To prove that $h_{2i}(X_2)$ depends on exactly one component of $Z$, exactly the same argument can be applied, replacing $({v},{y}, {\eta}, {\theta})$ with $({u},{t}, {\lambda}, {\mu})$, noting that the SDV assumption is also assumed for ${\lambda}$.
We thus see that 
	\begin{equation}
	\label{eq:u_onev}
	u_{i}({t})  = u_{i}(t_{\tilde{\pi}(i)})\,,
	\end{equation}
	where the permutation $\tilde{\pi}$ may be different from $\pi$.


We have shown that ${y}={h}_1({x}_1)$ and ${t}={h}_2({x}_2)$ estimate ${g}_1({z}, {n}_1)$ and ${g}_2({z}, {n}_2)$ up to two different gauges of all possible scalar invertible functions.
	
A remaining ambiguity could be that the two representations might be misaligned; that is, defining ${s}_1={g}_1({z}, {n}_1)$ and ${s}_2={g}_2({z}, {n}_2)$, while
	\begin{equation}
	s_{1,i} \independent s_{2,j} \forall i \neq j \label{eq:fact}
	\end{equation}
	we might have
	\[
	y_{\pi(i)} \independent t_{\tilde{\pi}(j)} \forall i \neq j\,,
	\]
	where $\pi(i)$, $\tilde{\pi}(i)$ are two different permutations of the indices $i=1, \ldots, n$. 
To show that this ambiguity is also resolved, we will show that
	\begin{equation}
	y_{i} \independent t_{j},\;\;\forall i \neq j. \label{eq:aim_lastpart}
	\end{equation}
	
	
	We recall that, by definition, we have $v_i(y_{\pi(i)}) = s_{1,i}$ and $u_j(t_{\tilde{\pi}(j)}) = s_{2,j}$. Then, due to equation \ref{eq:fact},
	\begin{align}
	v_i(y_{\pi(i)}) & \independent u_j(t_{\tilde{\pi}(j)}) \,\,\, \forall i \neq j \label{eq:permutind_1}\\
	\implies y_{\pi(i)} & \independent t_{\tilde{\pi}(j)} \,\,\, \forall i \neq j \label{eq:permutindep}\\
	\implies y_{i} & \independent t_{\tilde{\pi}\circ \pi^{-1} (j)} \,\,\, \forall i \neq j\,, \label{eq:permutind_2}
	\end{align}
	where the implication \ref{eq:permutind_1}-\ref{eq:permutindep} follows from invertibility of $v_i$ and $u_j$, and the implication \ref{eq:permutindep}-\ref{eq:permutind_2} follows from considering that, given that we know \ref{eq:permutindep}, we can define $l=\pi(j)$ and $k=\pi(i)$ and have
	\[
	y_{k}  \independent t_{\tilde{\pi} \circ \pi^{-1} (l)} \,\,\, \forall k \neq l.
	\]
	
	Define
	\[
	\tau = \tilde{\pi} \circ \pi^{-1}
	\]
	and note that it is a permutation. Then
	\begin{equation}
	y_i \independent t_{\tau(j)}  \forall i \neq j \label{eq:tauperm}
	\end{equation}
	
	Fix any particular $i$.
	Our goal is to show that for any $j\not= i$ the independence relation in Equation \ref{eq:aim_lastpart} holds.
	There are two possibilities:
	\begin{enumerate}
		\item $\tau(i)=i$
		\item $\tau(i)\neq i$
	\end{enumerate}
	In the first case, $\tau$ restricted to the set $\{1,\ldots,D\}\setminus\{i\}$ is still a permutation, and thus considering the independences of Equation \ref{eq:tauperm} for all $j\not= i$ implies each of the independences of Equation \ref{eq:aim_lastpart} and we are done.
	
	Let us consider the second case. Then,
	\[
	\exists l \in \{1, \ldots, D \}\setminus\{i\}\,\, \text{s.t.} \,\, l = \tau(i)\,.
	\]
	We then need to prove
	\begin{equation}
	y_i \independent t_l\,, \label{eq:ref_indices}
	\end{equation}
	
	which is the only independence implied by Equation \ref{eq:aim_lastpart} which is not implied by Equation \ref{eq:tauperm}.
	
	In order to do so, we rewrite equation \ref{eq:logistic}, yielding
	\begin{align}
\sum_{m}\psi_{m}(y_{m},t_{m}) \nonumber	= \sum_{m}\eta_{m}(v_m(y_{\pi(m)}), u_m(t_{\tilde{\pi}(m)}))	-\sum_{m}\theta_{i}(v_m(y_{\pi(m)}))
	\end{align}
	We now take derivative with respect to $y_i$ and $t_l$ in \ref{eq:ref_indices}; noting that $\tilde{\pi}^{-1}(l) = \pi^{-1}(i) $, we get
	\begin{align}
	0 = \frac{\partial^2}{\partial v_{\pi^{-1}(i)} \partial u_{\pi^{-1}(i)}} \eta_{\pi^{-1}(i)}(v_{\pi^{-1}(i)}(y_i), u_{\pi^{-1}(i)}(t_l)) \times \frac{\partial}{\partial y_i}v_{\pi^{-1}(i)}(y_i) \frac{\partial }{\partial t_l} u_{\pi^{-1}(i)}(t_l) \label{eq:perm_deriv}
	\end{align}
	
	Since $v_{\pi^{-1}(i)}(y_i)$ is a smooth and invertible function of its argument, the set of $y_i$ such that $\frac{\partial}{\partial y_i}v_{\pi^{-1}(i)}(y_i) = 0$ has measure zero.
	Similarly, $\frac{\partial }{\partial t_l} u_{\pi^{-1}(i)}(t_l) = 0$ on a set of measure zero.
	
	It therefore follows that
	\begin{align*}
	\frac{\partial}{\partial y_i}v_{\pi^{-1}(i)}(y_i) \frac{\partial }{\partial t_l} u_{\pi^{-1}(i)}(t_l) \neq 0
	\end{align*}
	almost everywhere and hence that
	\begin{equation}
	\frac{\partial^2}{\partial v_{\pi^{-1}(i)} \partial u_{\pi^{-1}(i)}} \eta_{\pi^{-1}(i)}(v_{\pi^{-1}(i)}(y_i), u_{\pi^{-1}(i)}(t_l)) = 0\,. \label{eq:additive_eta}
	\end{equation}
	almost everywhere.
	It thus follows that
	\begin{align*}
	\eta_{\pi^{-1}(i)}(v_{\pi^{-1}(i)}(y_i), u_{\pi^{-1}(i)}(t_l)) = \eta_{\pi^{-1}(i)}^y(v_{\pi^{-1}(i)}(y_i))+ \eta_{\pi^{-1}(i)}^t(u_{\pi^{-1}(i)}(t_l))
	\end{align*}
	which in turn implies that, for some functions $A$ and $B$, we can write
	\begin{align*}
	\log p(s_{1, \pi^{-1}(i)}|s_{2, \pi^{-1}(i)}) - \log p(s_{1, \pi^{-1}(i)}) = A(v_{\pi^{-1}(i)}(y_i)) + B(u_{\pi^{-1}(i)}(t_l))
	\end{align*}
	and therefore
	\begin{align*}
	\log p(s_{1, \pi^{-1}(i)},s_{2, \pi^{-1}(i)}) = C(v_{\pi^{-1}(i)}(y_i)) + D(u_{\pi^{-1}(i)}(t_l))
	\end{align*}
	for some functions $C$ and $D$. This decomposition of the log-pdf implies
	\begin{align*}
	s_{1, \pi^{-1}(i)} &\independent s_{2, \pi^{-1}(i)}\\
	\implies s_{1, \pi^{-1}(i)} &\independent s_{2, \tilde{\pi}^{-1}(l)}  \\
	\implies v_{\pi^{-1}(i)}(y_i)  &\independent u_{\tilde{\pi}^{-1}(l)}(t_l) \\
	\implies y_i  &\independent t_l \,,
	\end{align*}
	where the last implication holds due to invertibility of $v_{\pi^{-1}(i)}$ and $u_{\tilde{\pi}^{-1}(l)}$.
	
	This concludes the proof.
	
\end{proof}

\subsection{Proof of Corollary \ref{crl:lownoise}}
\label{appendix:thm2}


\begin{repcorollary}{crl:lownoise}
	Let $N_1^{(k)} = \frac{1}{k} \cdot  \Tilde{N}$ for $k \in \NN$, where $\Tilde{N}\in\mathbb{R}^D$ is a fixed random variable with finite variance, and let $N_2$ be a random variable that does not depend on $k$.
	Let $h_1^{(k)}, h_2^{(k)}$ be the output of the algorithm specified by Theorem \ref{thm:two-noisy-views} with noise variables $N_1^{(k)}$ and $N_2$.
	
	Suppose that the corrupters $g_i$ satisfy the following two criteria:
	\begin{enumerate}
		\item $\exists {a}  \in \mathbb{R}_{> 0}^D \: $   s.t. $\: \left|\frac{\partial g_1(z,n)}{\partial n} \right|_{n=0} \leq {a} \: $ for all ${z}$
		\item $\exists {b}  \in \mathbb{R}_{> 0}^D \: $ s.t. $\: 0<\frac{\partial g_1(z,0)}{\partial z} \leq b$
	\end{enumerate}
	Then, denoting by $\mathcal{E}$ the set of all component-wise, invertible functions, it holds that
	\[
	 \inf_{{e}\in \mathcal{E}}  \left \|Z - {e}(h_1^{(k)}(X_1)) \right \| \xrightarrow[k \to \infty]{p} 0
	\]
	where $p$ denotes convergence in probability.
\end{repcorollary}

\begin{proof}
To show that the random variable 
\begin{align*}
\inf_{{e}\in \mathcal{E}}  \left \|Z - {e}(h_1^{(k)}(X_1)) \right \|
\end{align*}
converges to $0$ in probability, we will prove the stronger statement that it converges in mean to $0$.
Denoting by ${d}^{(k)}_1$ the component-wise invertible ambiguity up to which ${g}(Z, N_1^{(k)})$ is recovered, we have that
	\begin{align}
	&\mathbb{E}_{Z, X_1} \left| \inf_{{e}\in \mathcal{E}}  \left \|Z - {e}(h_1^{(k)}(X_1)) \right \| \right| \\
	& = \mathbb{E}_{Z, X_1} \inf_{{e}\in \mathcal{E}}  \left \|Z - {e}(h_1^{(k)}(X_1)) \right \| \\
	& \leq \inf_{{e}\in \mathcal{E}} \mathbb{E}_{Z, X_1} \left \|Z - {e}(h_1^{(k)}(X_1)) \right \| \\
	& = \inf_{{e}\in \mathcal{E}} \mathbb{E}_{Z, N_1^{(k)}} \left \|Z - {e} \circ d_1^{(k)} \circ g_1(Z, N_1^{(k)}) \right \| \\
	& = \inf_{\tilde{e}\in \mathcal{E}} \mathbb{E}_{Z, N_1^{(k)}} \left \|Z - \tilde{e} \circ g_1(Z, N_1^{(k)}) \right \| \\
	& \leq \mathbb{E}_{Z, N_1^{(k)}} \left \|Z - e^* \circ g_1(Z, N_1^{(k)}) \right \| \label{eq:low_bounded} 
	\end{align}
	where the first upper bound holds by concavity, and the second holds for any ${e^*}\in\mathcal{E}$ by definition of infimum and in particular for ${e^*} = {g}_1 |^{-1}_{{n}=0}$, the existence of which is guaranteed by the assumptions on ${g}_1$.
	Taking a Taylor expansion of ${e^*} \circ  {g}_1(z, {n}_1^{(k)})$ around ${n}_1^{(k)}=0$ yields
	\begin{align*}
	&\mathbb{E}_{(Z, N_1^{(k)})} \Bigg[  \Bigg\|Z - e^* \circ {g}_1 (Z, 0) 
	+ \left.\left.\frac{\partial e^*}{\partial {g}_1} \frac{\partial {g}_1 (Z, 0)}{\partial {n}_1^{(k)}} \cdot N_1^{(k)} + \mathcal{O}(\|N_1^{(k)}\|^2) \right \| \right]\\
	&=\mathbb{E}_{(Z, N_1^{(k)})} \Bigg[  \Bigg\|\left.\left.\frac{\partial e^*}{\partial {g}_1} \frac{\partial {g}_1 (Z, 0)}{\partial {n}_1^{(k)}} \cdot N_1^{(k)} + \mathcal{O}(\|N_1^{(k)}\|^2) \right \| \right]\\
%	&\leq \mathbb{E}_{(Z, N_1^{(k)})} \left[  \left\| \frac{\partial e^*}{\partial {g}_1} \right\| \cdot \left\| \frac{\partial {g}_1 (Z, 0)}{\partial {n}_1^{(k)}} \right\| \cdot \left\| N_1^{(k)} \right\| + \left\| \mathcal{O}(\|N_1^{(k)}\|^2) \right \| \right]\\
%	&\leq \mathbb{E}_{(Z, N_1^{(k)})} \left[  \frac{b}{a} \left\| N_1^{(k)} \right\| + \left\| \mathcal{O}(\|N_1^{(k)}\|^2) \right \| \right]\\
	&\longrightarrow 0 \text{ as $k \longrightarrow \infty$}
	\end{align*}
	where the last equality follows from fact that $e^* = g |^{-1}_{\bm{n}=0}$ and the convergence follows from the fact that $N_1^{(k)} = \frac{1}{k} N_1$ where $N_1$ has finite variance (and thus mean) and from the boundedness conditions on the partial derivatives of $g_1$.
\end{proof}


\section{Proofs for multiple noisy views results}

\subsection{Proof of Lemma \ref{lem:last-lemma}}
\label{appendix:last-lemma}

\begin{replemma}{lem:last-lemma}
	Suppose that the sequence $\mathbb{E}_{N}[\Omega_{e}^M(Z, N)] = \frac{1}{M}\sum_{i=1}^M \mathbb{E}_{N_i}[{e}_i\circ {k}_i( Z + N_i)] $ converges as $M \to \infty$ for almost all $Z$, and write this limit as
	\begin{align*}
	\Omega_e(Z) = \lim_{M\to\infty}\mathbb{E}_{N}[\Omega_{e}^M(Z, N)].
	\end{align*}
	
	Suppose further that there exists $K$ such that $V_{e_i} = \mathrm{Var}\left({e}_i \circ {k}_i(Z + N_i) \right) \leq K$ for all $i$.
	Then
	\begin{align*}
	\Omega_{e}^M(Z, N) & \overset{a.s.}{\longrightarrow} \Omega_{e}(Z) \\
	R_{e, i}^M(Z, N) & \overset{a.s.}{\longrightarrow} R_{e, i}(Z, N_i) = {e}_i\circ {k}_i( Z + N_i) - \Omega_{e}(Z)
	\end{align*}
\end{replemma}

In proving this, we will make crucial use of \emph{Kolmogorov's strong law}:

\medskip

\begin{theorem}
	Suppose that $X_m$ is a sequence of independent (but not necessarily identically distributed) random variables with
	\begin{align*}
	\sum_{m=1}^\infty \frac{1}{m^2}\mathrm{Var} [X_m] < \infty
	\end{align*}
	Then,
	\begin{align*}
	\frac{1}{M}\sum_{m=1}^M X_m - \mathbb{E}[X_m] \overset{a.s.}{\longrightarrow} 0
	\end{align*}
\end{theorem}

\begin{proof}[Proof of Lemma \ref{lem:last-lemma}]
Fix ${z}$ and consider $\Omega_{e}^M({z}, {n})$ as a random variable with randomness induced by ${n}=(n_1,n_2,\ldots)$.
We will show that for almost all ${z}$ this converges ${n}$-almost surely to a constant, and hence $\Omega_{e}^M({z}, {n})$ converges almost surely to a function of ${z}$.

The law of total expectation says that
\begin{align*}
&\mathrm{Var}_{{z}, {n}_i} [{e}_i\circ {k}_i({z} + {n}_i)] \\
&= \mathbb{E}_{{z}}\left[ V_i({z}) \right] + \mathrm{Var}_{{z}}\left[ \mathbb{E}_{{n}_i} [{e}_i\circ {k}_i( {z} + {n}_i)] \right] \\
& \geq \mathbb{E}_{{z}}\left[ V_i({z}) \right].
\end{align*}
Since by assumption $\mathrm{Var}_{{z}, {n}_i} [{e}_i\circ {k}_i({z} + {n}_i)] \leq K$, we have that
\begin{align*}
\mathbb{E}_{{z}}\left[ \sum_{i=1}^\infty  \frac{V_i({z})}{i^2} \right] \leq \frac{ K \pi^2}{6}
\end{align*}
and therefore  $\sum_{i=1}^\infty  \frac{V_i({z})}{i^2} < \infty$ with probability $1$ over ${z}$, else the expectation above would be unbounded since $V_i({z})\geq 0$.

We have further that for almost all ${z}$,
\begin{align*}
\Omega_{{e}}({z}) = \lim_{M\to\infty}\frac{1}{M}\sum_{i=1}^M E_{{e}_i}({z})
\end{align*}
exists.
Therefore, for almost all $s$ the conditions of Kolmogorov's strong law are met by $\Omega_{{e}}^M({z}, {n})$ and so
\begin{align*}
\Omega_{{e}}^M({z}, {n}) - \mathbb{E}_{{n}}[\Omega_{{e}}^M({z}, {n})] \overset{{n}-a.s.}{\longrightarrow} 0
\end{align*}

Since $\mathbb{E}_{{n}}[\Omega_{{e}}^M({z}, {n})] \overset{{n}-a.s.}{\longrightarrow} \Omega_{{e}}({z})$, it follows that
\begin{align*}
\Omega_{{e}}^M({z}, {n}) \overset{{n}-a.s.}{\longrightarrow} \Omega_{{e}}({z}).
\end{align*}
Since this holds with probability $1$ over ${z}$, we have that
\begin{align*}
\Omega_{{e}}^M({z}, {n}) \overset{{n}-a.s.}{\longrightarrow} \Omega_{{e}}({z}).
\end{align*}

It follows that we can write

\begin{align*}
R_{{e}, i}^M({z}, {n}) &= {e}_i\circ {k}_i( {z} + {n}_i) - \Omega_{{e}}^M({z}, {n}) \\
&\overset{a.s.}{\longrightarrow} R_{{e}, i}({z}, {n}_i):= {e}_i\circ {k}_i( {z} + {n}_i) - \Omega_{{e}}({z})
\end{align*}
\end{proof}

\subsection{Proof of Theorem \ref{thm:lastthm}}
\label{sec:lasttmpr}


\begin{reptheorem}{thm:lastthm}
	Suppose there exists $C>0$ such that $\text{Var}(N_i) \leq C$ for all $i$ and let $\mathcal{G}_K = \big\lbrace
	\{{e}_i \}$ s.t.
	\begin{align}
	& V_{{e}_i} \leq K \ \forall i \tag{\ref{eq:resid_1}}\\
	& \Omega_{{e}}(Z) < \infty \  \text{ for almost all } Z \tag{\ref{eq:resid_2}}\\
	&R_{{e}, i} \independent R_{{e}, j} \ \forall i \not= j, \tag{\ref{eq:resid_3}}\\
	\nonumber \\    &\mathbb{E} R_{{e}, i} = 0 \ \forall i \tag{\ref{eq:resid_5}} \\
	&R_{{e}, i}(Z, N_i) = R_{{e}, i}(N_i) \ \forall i \ \big\rbrace \tag{\ref{eq:resid_6}}
	\end{align}
	
	Then,
	\begin{align*}
	\mathcal{G}_K \subseteq\left\lbrace \{ {\alpha} {k}^{-1}_i + {\beta} \} \ : \ {\alpha} \in \mathbb{R}^{D}_{\not=0}, \: {\beta} \in \mathbb{R}^{D} \right\rbrace
	\end{align*}
	where $\alpha {k}^{-1}_i$ denotes the element-wise product with the scalar elements of ${\alpha}$.
	If $K \geq \text{Var}(Z) + C$, then $ \{ {k}^{-1}_i \}  \in \mathcal{G}_K$,
	and so $\mathcal{G}_K$ is non-empty for $K$ sufficiently large.
\end{reptheorem}

We will begin by showing that if $K \geq \mathrm{Var}({z}) + C$ then $\{ {k}^{-1}_i \}  \in \mathcal{G}_K$.

For ${e}_i = {k}_i^{-1}$, we have that
\begin{align*}
\Omega_{{e}}^M({z}, {n}) = \frac{1}{M} \sum_{i=1}^M {z} + {n}_i &\overset{a.s.}{\longrightarrow} {z} = \Omega_{{e}}^M({z})\\
R_i^M = {z} + {n}_i - \Omega_{{e}}({z}, {n})  &\overset{a.s.}{\longrightarrow} {n}_i = R_{{e}, i}({n}_i)
\end{align*}
where the convergences follow from application of Kolmogorov's strong law, using the fact that $\mathrm{Var}({n}_i) \leq C$ for all $i$.
Satisfaction of condition \ref{eq:resid_1} follows from the fact that $\mathrm{Var}_{{z}, {n}_i} ({z} + {n}_i) \leq C + \mathrm{Var}({z}) \leq K$.
Since ${z}$ is a well-defined random variable,  $\Omega_{{e}}({z}) < \infty$ with probability $1$, satisfying condition \ref{eq:resid_2}.
It follows from the mutual independence of ${n}_i$ and ${n}_j$ that $R_{{e}, i}$ and $R_{{e}, j}$ satisfy condition \ref{eq:resid_3}.
Condition \ref{eq:resid_5} follows from the fact that $\mathbb{E}[{n}_i]=0$
Condition \ref{eq:resid_6} follows from $R_{{e}, i}$ being constant as a function of ${z}$.

It therefore follows that $\{ {k}^{-1}_i \}  \in \mathcal{G}_K$ for $K$ sufficiently large.


We will next show that if $\{ {e}_i\} \in \mathcal{G}_K$ then there exist a matrix ${\alpha}$ and vector ${\beta}$ such that ${e}_i = {\alpha} {k}_i^{-1} + {\beta}$ for all $i$.
Since ${e}_i$ acts coordinate-wise, it moreover follows that ${\alpha}$ is diagonal.

First, we will show that each ${e}_i\circ{k}_i$ is affine, i.e. there exist potentially different ${\alpha}_i, {\beta}_i$ such that ${e}_i = {\alpha}_i {k}_i^{-1} + {\beta}_i$ for each $i$.

Then we will show that we must have ${\alpha}_i = {\alpha}_j$ and ${\beta}_i = {\beta}_j$ for all $i,j$.

To see that ${e}_i$ is affine, we make use of that fact that $R_{{e},i}$ is constant as a function of ${z}$.
It follows that for any $x$ and $y$
\begin{align*}
{e}_i\circ{k}_i(x + y) &= R_{{e},i}(x) + \Omega_{{e}}(y) \\
&= R_{{e},i}(x) + \Omega_{{e}}(0) + R_{{e},i}(0) + \Omega_{{e}}(y) \\
& \qquad- \left(R_{{e},i}(0) +  \Omega_{{e}}(0)\right) \\
&= {e}_i\circ{k}_i(x) + {e}_i\circ{k}_i(y) - {e}_i\circ{k}_i(0)
\end{align*}
It therefore follows that ${e}_i\circ{k}_i$ is affine, since if we define
\begin{align*}
L(x + y) &= {e}_i\circ{k}_i(x + y) - {e}_i\circ{k}_i(0) \\
&= \left({e}_i\circ{k}_i(x) - {e}_i\circ{k}_i(0)\right) \\
& \qquad+ \left({e}_i\circ{k}_i(y) - {e}_i\circ{k}_i(0)\right) \\
&= L(x) + L(y)
\end{align*}
then $L$ is linear and we can write ${e}_i\circ{k}_i(x)$ as the sum of a linear function and a constant:
\begin{align*}
{e}_i\circ{k}_i(x) = L(x) + {e}_i\circ{k}_i(0)
\end{align*}
Thus ${e}_i\circ{k}_i$ is affine, and we have some (diagonal) matrix ${\alpha}_i$ and vector ${\beta}_i$ such that for any $x$
\begin{align*}
&{e}_i\circ{k}_i(x) = {\alpha}_i x  + {\beta}_i \\
\implies& {e}_i \left(x \right) = {\alpha}_i {k}_i^{-1} x + {\beta}_i.
\end{align*}



Next we show that for the set of $\{{e}_i = {\alpha}_i {k}_i^{-1} + {\beta}_i\}$, it must be the case that each ${\alpha}_i = {\alpha}_j$ and ${\beta}_i = {\beta}_j$.

Observe that
\begin{align*}
\Omega_{{e}}^M({z}, {n}) &= \frac{1}{M} \sum_{i=1}^M {\alpha}_i {z} + {\alpha}_i {n}_i + {\beta}_i \\
&= \left( \frac{1}{M} \sum_{i=1}^M {\alpha}_i \right){z} + \frac{1}{M} \sum_{i=1}^M {\beta}_i  + \frac{1}{M} \sum_{i=1}^M {\alpha}_i {{n}}_i \\
\mathbb{E}_{{n}}[\Omega_{{e}}^M({z}, {n})] &= \left( \frac{1}{M} \sum_{i=1}^M {\alpha}_i \right){z} +  \frac{1}{M} \sum_{i=1}^M {\beta}_i
\end{align*}

Define
\begin{align*}
{\alpha} &= \lim_{M\to\infty}\frac{1}{M}\sum_{i=1}^M {\alpha}_i \\
{\beta} &= \lim_{M\to\infty}\frac{1}{M}\sum_{i=1}^M {\beta}_i \\
\end{align*}
which exist by the assumption that $\Omega_{{e}}^M({z}, {n})$ converges as $M\to\infty$.
Thus
\begin{align*}
\Omega_{{e}}({z}) &= {\alpha} {z} + {\beta} \\
R_{{e}, i}({z}, {n}_i) &= ({\alpha}_i - {\alpha}){z} + {\alpha}_i{n}_i + {\beta}_i - {\beta}
\end{align*}
Now, suppose that there exist $i$ and $j$ such that such that ${\alpha}_i \not= {\alpha}_j$.
It follows that
\begin{align*}
R_{{e}, i}({z}, {n}_i) &= ({\alpha}_i - {\alpha}){z} + {\alpha}_i{n}_i + {\beta}_i - {\beta} \\
R_{{e}, j}({z}, {n}_j) &= ({\alpha}_j - {\alpha}){z} + {\alpha}_j{n}_j + {\beta}_j - {\beta}
\end{align*}
There are two cases.
If ${\alpha}_i \not={\alpha}$, then $R_{{e}, i}({z}, {n}_i)$ is not a constant function of ${z}$.
But if ${\alpha}_i ={\alpha}$, then ${\alpha}_j \not={\alpha}$ and so $R_{{e}, j}({z}, {n}_j)$ is not a constant function of ${z}$.
This is a contradiction, and so ${\alpha}_i = {\alpha}_j$ for all $i,j$.

Suppose similarly that there exist ${\beta}_i \not={\beta}_j$.
If ${\beta}_i \not={\beta}$, then $\mathbb{E}[R_{{e}, i}({n}_i)] = {\beta}_i - {\beta}$ which is non-zero.
If ${\beta}_i ={\beta}$, then ${\beta}_j \not={\beta}$ and so $\mathbb{E}[R_{{e}, j}({n}_j)] = {\beta}_j - {\beta}$ is non-zero.
This is a contradiction, and so ${\beta}_i = {\beta}_j$ for all $i,j$.

We have thus proven that set $\{ {e}_i \} \in \mathcal{G}_K$ is of the form ${e}_i = {\alpha} {k}_i^{-1} + {\beta}$ for all $i$.