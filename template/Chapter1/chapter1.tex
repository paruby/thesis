%!TEX root = ../thesis.tex
%*******************************************************************************
%*********************************** First Chapter *****************************
%*******************************************************************************

\chapter{Introduction}  %Title of the First Chapter

\ifpdf
    \graphicspath{{Chapter1/Figs/Raster/}{Chapter1/Figs/PDF/}{Chapter1/Figs/}}
\else
    \graphicspath{{Chapter1/Figs/Vector/}{Chapter1/Figs/}}
\fi


% Example nomenclature definitions

\nomenclature[z-cif]{$CIF$}{Cauchy's Integral Formula}                                % first letter Z is for Acronyms 
\nomenclature[a-F]{$F$}{complex function}                                                   % first letter A is for Roman symbols
\nomenclature[g-p]{$\pi$}{ $\simeq 3.14\ldots$}                                             % first letter G is for Greek Symbols
\nomenclature[g-i]{$\iota$}{unit imaginary number $\sqrt{-1}$}                      % first letter G is for Greek Symbols
\nomenclature[g-g]{$\gamma$}{a simply closed curve on a complex plane}  % first letter G is for Greek Symbols
\nomenclature[x-i]{$\oint_\gamma$}{integration around a curve $\gamma$} % first letter X is for Other Symbols
\nomenclature[r-j]{$j$}{superscript index}                                                       % first letter R is for superscripts
\nomenclature[s-0]{$0$}{subscript index}                                                        % first letter S is for subscripts

\begin{itemize}
  \item Non-technical introduction that should be accessible to people who don't know about machine learning, e.g. my parents.
  \begin{itemize}
  \item Describe machine learning as way to program computers.
  \item When humans look at a picture, we don't see pixels. We immediately see higher level concepts.
  \item An important topic in ML, and the subject of this thesis, is, roughly speaking, how machines can learn high level concepts. 
  \item The example of images is easy to grasp because we are familiar with the idea of objects like cats and dogs. In fact, the difficult thing to understand is that images (on a screen) are fundamentally an array of numbers. 
  \item Another example: audio. If I showed you a picture of a wave form, you wouldn't understand what it is. But if I play it to you, you'd be able to decompose the continuous stream into different parts (voice, drums, ...)
  \item One of the key features of human perception is the ability to understand the world at different scales of detail. For instance, (image of car) is at its simplest just a car. But a car consists of doors, windows, a wind shield, wheels. Each of these components can be more closely inspected: each wheel has the metal central part and rubber tyres, and we know that out of view there is a complicated steering mechanism connects the wheels to the rest of the car. If we inspected the tyres closely we might have interesting things to say about the tread, and so on. Similarly, an album of music consists of songs that are related, each song consists of chorus and verse, within each of these there is a progression of chords, and so on. Chapter 3 presents the first major topic of my PhD, which considers how to mathematically describe the fact that there is no one objective level at which we understand any system; rather, we are aware that any understanding exists at some particular scale, and that depending on what we are doing or trying to achieve, thinking in more or less detailed ways may be appropriate.
  \item Another feature of human perception is our ability to synthesise together different streams of perceptual information into one conscious experience. For example, each of our eyes sees a 2D image. Yet we perceive the world in 3D because our brains automatically merge these two distinct streams together. (This is chapter 4, ICA stuff)
  \item Think about how to describe autoencoder and learning theory stuff
  \end{itemize}
\end{itemize}

\section{Outline}

\section{Contributions}

\section{Summary of PhD work not included in this thesis}





















