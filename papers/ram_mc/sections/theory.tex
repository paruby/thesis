\section{Random mixture estimator and convergence
results}\label{sec:theory}
In this section we introduce our $f$-divergence estimator, and present theoretical guarantees for it.
We assume the existence of probability distributions
${P_Z}$ and ${Q_Z}$ defined over $\Z$ with known density $p(z)$ and intractable density ${q(z) = \int q(z|x) q(x) dx}$ respectively,  where ${Q_{Z|X}}$ is known. $Q_X$ defined over $\X$ is unknown, however we have an i.i.d.\:sample ${\XN=\{X_1, \ldots, X_N\}}$ from it.
Our ultimate goal is to estimate the intractable $f$-divergence $D_f(Q_Z \| P_Z)$ defined by:
\begin{definition}[$f$-divergence]
\label{def:fdiv}
Let $f$ be a convex function on $(0, \infty)$ with $f(1) = 0$. 
The $f$-divergence $D_f$ between distributions $Q_Z$ and $P_Z$ admitting densities $q(z)$ and $p(z)$ respectively is
{\addtolength{\abovedisplayskip}{-0.5mm}
\addtolength{\belowdisplayskip}{-0.5mm}
\begin{align*}
    D_f(Q_Z \| P_Z) := \int f \left( \frac{q(z)}{p(z)} \right) p(z) dz.
\end{align*}}%
\end{definition}
Many commonly used divergences such as Kullback–Leibler and $\chi^2$ are $f$-divergences.
All the divergences considered in this paper together with their corresponding $f$ can be found in Appendix~\ref{appendix:f-fns}. 
Of them, possibly the least well-known in the machine learning literature are $f_\beta$-divergences \cite{osterreicher2003new}. 
These symmetric divergences are continuously parameterized by $\beta\in(0, \infty]$. Special cases include squared-Hellinger ($\mathrm{H}^2$) for ${\beta=\frac{1}{2}}$,  Jensen-Shannon (JS) for $\beta=1$, Total Variation (TV) for $\beta=\infty$. 

In our setting $Q_Z$ is intractable and so is ${D_f(Q_Z \| P_Z)}$.
Substituting $Q_Z$ with a sample-based finite mixture ${\hat{Q}_Z^N := \frac{1}{N} \sum_{i=1}^N Q_{Z|X_i}}$ leads to our proposed 
{\bf Random Mixture estimator (RAM)}:
{\addtolength{\abovedisplayskip}{-1mm}
\addtolength{\belowdisplayskip}{-1mm}
\begin{equation}\textstyle
    D_f\bigl(\hat{Q}_Z^N \| P_Z\bigr) := D_f\Big(\frac{1}{N} \sum_{i=1}^N Q_{Z|X_i} \big\| P_Z\Big).  % \approx D_f(Q_Z \| P_Z).
\end{equation}}%
Although $\smash{\hat{Q}_Z^N}$ is a function of $\smash{\XN}$ we omit this dependence in notation for brevity. 
In this section we identify sufficient conditions under which $\smash{D_f(\hat{Q}_Z^N \| P_Z)}$ is a ``good'' estimator of $\smash{D_f(Q_{Z} \| P_Z)}$.
More formally, we establish conditions under which the estimator is asymptotically unbiased, concentrates to its expected value and can be practically estimated using Monte-Carlo sampling.

\subsection{Convergence rates for the bias of RAM}
The following proposition shows that $D_f(\hat{Q}_Z^N \| P_Z)$ upper bounds $D_f(Q_{Z} \| P_Z)$ in expectation for any finite $N$, and that the upper bound becomes tighter with increasing $N$:
%
\begin{proposition}\label{prop:upper-bound}
Let $M \leq N$ be integers. Then
\begin{align}
\label{eq:our-estimate}
    D_f(Q_Z \| P_Z) \ \leq 
    \mathbb{E}_{\mathbf{X}^N} \bigl[D_f(\hat{Q}_Z^N \| P_Z)\bigr] \  \leq \ \mathbb{E}_{\mathbf{X}^M} \bigl[D_f(\hat{Q}_Z^M \| P_Z)\bigr].
\end{align}
\end{proposition}
\begin{proof}[Proof sketch (full proof in Appendix \ref{proof:prop1})]
The first inequality follows from Jensen's inequality, using the facts that $f$ is convex and ${Q_Z = \E_{\XN} [\hat{Q}_Z^N}]$.
The second holds since a sample ${\XM}$ can be drawn by sub-sampling (without replacement) $M$ entries of ${\XN}$, and by applying Jensen again.
\end{proof}
As a function of $N$, the expectation is a decreasing sequence that is bounded below.
By the monotone convergence theorem, the sequence converges.
Theorems \ref{thm:fast-KL-rate} and \ref{thm:convergence-rate-general} in this section give sufficient conditions under which the expectation of RAM converges to $D_f(Q_{Z} \| P_Z)$ as $N\to\infty$ for a variety of $f$ and provide rates at which this happens, summarized in Table \ref{table:convergence}.
The two theorems are proved using different techniques and assumptions. 
These assumptions, along with those of existing methods (see Table~\ref{table:convergence-other}) are discussed at the end of this section.

\renewcommand{\arraystretch}{1}
\begin{table}
 \caption{Rate of bias $\E_{\XN} D_f\big(\hat{Q}^N_{Z} \| P_Z\big) - D_f\left(Q_{Z} \| P_Z\right)$.}
 \label{table:convergence}
 \centering
 \begin{tabular}{c c c c c c c c c } 
 \toprule
 \multirow{2}{*}{$f$-divergence} & \multirow{2}{*}{KL} & \multirow{2}{*}{TV} & \multirow{2}{*}{$\chi^2$} & \multirow{2}{*}{$\text{H}^2$} & \multirow{2}{*}{JS} & \multicolumn{2}{c}{\thead{$D_{f_\beta}$}}  & \thead{$D_{f_\alpha}$} \\ [-0.8ex]
 & & & & & & $\scriptstyle{\frac{1}{2}<\beta<1}$ & $\scriptstyle{1<\beta<\infty}$ &
$\scriptstyle{-1<\alpha<1}$ \\
 \midrule
 \thead{Theorem 1} & $\scriptstyle{N^{-1}}$ & $\scriptstyle{N^{-\frac{1}{2}}}$ & - & $\scriptstyle{N^{-\frac{1}{2}}}$ & $\scriptstyle{N^{-\frac{1}{4}}}$ & $\scriptstyle{N^{-\frac{1}{4}}}$ & $\scriptstyle{N^{-\frac{1}{4}}}$ & - \\ 
 \thead{Theorem 2} & $\scriptstyle{N^{-\frac{1}{3}}\log N}$ & $\scriptstyle{N^{-\frac{1}{2}}}$ & $\scriptstyle{N^{-1}}$ & $\scriptstyle{N^{-\frac{1}{5}}}$ & $\scriptstyle{N^{-\frac{1}{3}}\log N}$ & $\scriptstyle{N^{-\frac{1}{3}}}$ & $\scriptstyle{N^{-\frac{1}{2}}}$ & $\scriptstyle{N^{-\frac{\alpha+1}{\alpha+5}}}$ \\
 \bottomrule
\end{tabular}
\end{table}

\begin{theorem}[Rates of the bias]\label{thm:fast-KL-rate}
If
$\E_{X\sim Q_X}\bigl[\chi^2\bigl(Q_{Z|X}, Q_Z\bigr)\bigr]$ and
$\KL\left( Q_{Z} \| P_Z\right)$ are finite then the bias ${\E_{\XN}\bigl[D_f( \hat{Q}_Z^N \| P_Z)\bigr] - D_f\left( Q_{Z} \| P_Z\right)}$ decays with rate as given in the first row of Table~\ref{table:convergence}.
\end{theorem}
\begin{proof}[Proof sketch (full proof in Appendix \ref{appendix:subsec:thm1})]
There are two key steps to the proof. 
The first is to bound the bias by ${\E_{\XN}\big[D_f(\hat{Q}_Z^N, Q_Z)\big]}$. 
For the KL this is an equality. 
For ${D_{f_\beta}}$ this holds because for $\beta {\geq} 1/2$ it is a \emph{Hilbertian metric} and its square root satisfies the triangle inequality \citep{hein05hilbertian}.
The second step is to bound ${\E_{\XN}\bigl[D_f(\hat{Q}_Z^N, Q_Z)\bigr]}$ in terms of ${\E_{\XN}\bigl[\chi^2(\hat{Q}_Z^N, Q_Z)\bigr]}$, which is the variance of the average of $N$ i.i.d.\:random variables and therefore decomposes as ${\E_{X\sim Q_X}\bigl[\chi^2(Q_{Z|X}, Q_Z)\bigr] / N}$.
\end{proof}
%
\begin{theorem}[Rates of the bias]\label{thm:convergence-rate-general}
If $\E_{X\sim Q_X, Z\sim P_Z}\bigl[ q^4(Z|X) / p^4(Z) \bigr]$ is finite then
the bias $\E_{\XN}\bigl[D_f( \hat{Q}_Z^N \| P_Z)\bigr] - D_f\left( Q_{Z} \| P_Z\right)$ decays with rate as given in the second row of Table \ref{table:convergence}.
\end{theorem}
\begin{proof}[Proof sketch (full proof in Appendix \ref{proof:thm2})]
Denoting by $\hat{q}_N(z)$ the density of $\hat{Q}_Z^N$,
the proof is based on the inequality
$f\bigl(\hat{q}_N(z) / p(z)\bigr) - f\bigl(q(z) / p(z)\bigr)\leq \frac{\hat{q}_N(z) - q(z)}{p(z)} f'\bigl(\hat{q}_N(z) / p(z)\bigr)$ due to convexity of $f$, applied to the bias.
The integral of this inequality is bounded by controlling $f'$, requiring subtle treatment when $f'$ diverges when the density ratio $\hat{q}_N(z)/p(z)$ approaches zero.
\end{proof}

\subsection{Tail bounds for RAM and practical estimation with RAM-MC}

Theorems \ref{thm:fast-KL-rate} and \ref{thm:convergence-rate-general} describe the convergence of the  \emph{expectation} of RAM over $\XN$, which in practice may be intractable.
Fortunately, the following shows that RAM rapidly concentrates to its expectation.


\begin{table}
 \caption{Rate $\psi(N)$ of high probability bounds for $D_f\big(\hat{Q}^N_{Z} \| P_Z\big)$ (Theorem 3).}
 \label{table:concentration}
 \centering
 \begin{tabular}{c c c c c c c c c } 
 \toprule
 \multirow{2}{*}{$f$-divergence} & \multirow{2}{*}{KL} & \multirow{2}{*}{TV} & \multirow{2}{*}{$\chi^2$} & \multirow{2}{*}{$\text{H}^2$} & \multirow{2}{*}{JS} & \multicolumn{2}{c}{\thead{$D_{f_\beta}$}}  & \thead{$D_{f_\alpha}$} \\ [-0.8ex]
 & & & & & & $\scriptstyle{\frac{1}{2}<\beta<1}$ & $\scriptstyle{1<\beta<\infty}$ &
$\scriptstyle{\frac{1}{3}<\alpha<1}$ \\
 \midrule
 \thead{$\psi(N)$} &  $\scriptstyle{N^{-\frac{1}{6}}\log N}$ & $\scriptstyle{N^{-\frac{1}{2}}}$ & 
 $\scriptstyle{N^{-\frac{1}{2}}}$ &
 - & 
 $\scriptstyle{N^{-\frac{1}{6}}\log N}$ &
 $\scriptstyle{N^{-\frac{1}{6}}}$ &
 $\scriptstyle{N^{-\frac{1}{2}}}$ &
 $\scriptstyle{N^{\frac{1-3\alpha}{\alpha+5}}}$
 \\ 
 \bottomrule
\end{tabular}
\end{table}

\begin{theorem}[Tail bounds for RAM]\label{thm:concentration}
Suppose that ${\chi^2\left(Q_{Z|x} \| P_Z\right) \leq C < \infty}$ for all $x$ and for some constant $C$.
Then, the RAM estimator ${D_f( \hat{Q}_Z^N \| P_Z)}$ concentrates to its mean in the following sense. 
For $N>8$ and for any $\delta >0$, with probability at least $1-\delta$ it holds that
\begin{align*}
    \left| D_f( \hat{Q}_Z^N \| P_Z) - \mathbb{E}_{\XN} \bigl[D_f(\hat{Q}_Z^N \| P_Z)\bigr] \right| \leq {K \cdot \psi(N)} \  \sqrt{\log (2/\delta)},
\end{align*}
where $K$ is a constant and $\psi(N)$ is given in Table~\ref{table:concentration}.
\end{theorem}
\begin{proof}[Proof sketch (full proof in Appendix \ref{proof:thm3})]
These results follow by applying McDiarmid's inequality.
To apply it we need to show that RAM viewed as a function of $\XN$ has bounded differences.
We show that when replacing $\smash{X_i\in\XN}$ with $\smash{X_i'}$ the value of $\smash{D_f( \hat{Q}_Z^N \| P_Z)}$ changes by at most $\smash{O(N^{-1/2}\psi(N))}$.
Proof of this proceeds similarly to the one of Theorem \ref{thm:convergence-rate-general}.
\end{proof}

In practice it may not be possible to evaluate $\smash{D_f( \hat{Q}_Z^N \| P_Z)}$ analytically. 
We propose to use Monte-Carlo (MC) estimation since both densities $\hat{q}_N(z)$ and $p(z)$ are assumed to be known.
We consider importance sampling with proposal distribution ${\pi(z|\XN)}$, highlighting the fact that $\pi$ can depend on the sample $\XN$.
If $\pi(z|\XN) = p(z)$ this reduces to normal MC sampling. 
We arrive at the {\bf RAM-MC estimator} based on $M$ i.i.d.\:samples $\ZM:=\{Z_1,\dots,Z_M\}$ from $\pi(z|\XN)$:
\begin{align}
\label{eq:our-mc-estimate}
    %\phi_\pi\left(\ZM, \XN\right) := 
    \hat{D}^M_f( \hat{Q}_Z^N \| P_Z) :=
    \frac{1}{M}\sum_{m=1}^M f\left( \frac{\hat{q}_N(Z_m)}{p(Z_m)} \right) \frac{p(Z_m)}{\pi\left(Z_m|\XN\right)}.
\end{align}

\begin{theorem}[RAM-MC is unbiased and consistent]\label{thm:mc-variance}
$\E\bigl[\hat{D}^M_f( \hat{Q}_Z^N \| P_Z)\bigr]
=\E\bigl[D_f( \hat{Q}^N_{Z} \| P_Z )\bigr]$ for any proposal distribution $\pi$.
If $\pi(z|\XN) = p(z)$ or $\pi(z | \XN) = \hat{q}_N(z)$ then under mild assumptions$^\star$ on the moments of $q(Z|X)/p(Z)$
and denoting by ${\psi(N)}$ the rate given in Table~\ref{table:concentration}, we have
\begin{align*}
    \text{Var}_{\XN, \ZM} \bigl[\hat{D}^M_f( \hat{Q}_Z^N \| P_Z)\bigr] = 
    O\left(M^{-1}\right) + O\left( \psi(N)^2 \right).
\end{align*}
\end{theorem}
\begin{proof}[Proof sketch ($^\star$full statement and proof in Appendix \ref{appendix:full-statment-proof-mc})]
By the law of total variance, 
\begin{align*}
    \text{Var}_{\XN, \ZM} \bigl[\hat{D}^M_f\bigr] = 
    \mathbb{E}_{\XN} \bigl[\text{Var}\bigl[\hat{D}^M_f\, | \XN\bigr]\bigr] + \text{Var}_{\XN} \bigl[D_f( \hat{Q}_Z^N \| P_Z )\bigr].
\end{align*}
The first of these terms is ${O( M^{-1})}$ by standard results on MC integration, subject to the assumptions on the moments.
Using the fact that ${\text{Var}[Y] = \int_0^\infty\mathbb{P} ( |Y - \mathbb{E} Y| > \sqrt{t}) dt}$ for any random variable $Y$
we bound the second term by integrating the exponential tail bound of Theorem~\ref{thm:concentration}.
\end{proof}

Through use of the Efron-Stein inequality---rather than integrating the tail bound provided by McDiarmid's inequality---it is possible for some choices of $f$ to weaken the assumptions under which the $O(\psi(N)^2)$ variance is achieved: from uniform boundedness of $\smash{\chi^2(Q_{Z|X}\|P_Z)}$ to boundedness in expectation.
In general, a variance better than ${O(M^{-1})}$ is not possible using importance sampling. However, the constant and hence practical performance may vary significantly depending on the choice of $\pi$.
We note in passing that through Chebyshev's inequality, it is possible to derive confidence bounds for RAM-MC of the form similar to Theorem~\ref{thm:concentration}, but with an additional dependence on $M$ and worse dependence on $\delta$. 
For brevity we omit this.


\renewcommand{\arraystretch}{1}
\begin{table}
 \caption{Rate of bias for other estimators of $D_f(P,Q)$.}
 \label{table:convergence-other}
 \centering
 \begin{tabular}{c c c c c c c c c } 
 \toprule
 \multirow{2}{*}{$f$-divergence} & \multirow{2}{*}{KL} & \multirow{2}{*}{TV} & \multirow{2}{*}{$\chi^2$} & \multirow{2}{*}{$\text{H}^2$} & \multirow{2}{*}{JS} & \multicolumn{2}{c}{\thead{$D_{f_\beta}$}}  & \thead{$D_{f_\alpha}$} \\ [-0.8ex]
 & & & & & & $\scriptstyle{\frac{1}{2}<\beta<1}$ & $\scriptstyle{1<\beta<\infty}$ &
$\scriptstyle{-1<\alpha<1}$ \\
 \midrule
 \thead{Krishnamurthy et al. [22]} & - & - & - & - & - & - & - & $\scriptstyle{N^{-\frac{1}{2}} + N^{\frac{-3s}{2s + d}}}$ \\ 
 \thead{Nguyen et al. [28]} & $\scriptstyle{N^{-\frac{1}{2}}}$ & - & - & - & - & - & - & - \\ 
 \thead{Moon and Hero [26]} & $\scriptstyle{N^{-\frac{1}{2}}}$ & - & $\scriptstyle{N^{-\frac{1}{2}}}$ & $\scriptstyle{N^{-\frac{1}{2}}}$ & $\scriptstyle{N^{-\frac{1}{2}}}$ & $\scriptstyle{N^{-\frac{1}{2}}}$ & $\scriptstyle{N^{-\frac{1}{2}}}$ & $\scriptstyle{N^{-\frac{1}{2}}}$ \\ 
 \bottomrule
\end{tabular}
\end{table}

\subsection{Discussion: assumptions and summary}\label{subsection:discussion-assumptions}
All the rates in this section are independent of the dimension of the space $\mathcal{Z}$ over which the distributions are defined.
However the constants may exhibit some dependence on the dimension.
Accordingly, for fixed $N$, the bias and variance may generally grow with the dimension.

Although the data distribution $Q_X$ will generally be unknown, in some practical scenarios such as deep autoencoder models, $P_Z$ may be chosen by design and $Q_{Z|X}$ learned subject to architectural constraints.
In such cases, the assumptions of Theorems \ref{thm:convergence-rate-general} and \ref{thm:concentration} can be satisfied by making suitable restrictions (we conjecture also for Theorem~\ref{thm:fast-KL-rate}).
For example, suppose that ${P_Z}$ is  ${\mathcal{N}\left(0, I_d\right)}$ and ${Q_{Z|X}}$ is  ${\mathcal{N}\left( \mu(X), \Sigma(X)\right)}$ with $\Sigma$ diagonal. 
Then the assumptions hold if there exist constants $K, \epsilon > 0$ such that ${\| \mu(X)\| < K}$ and ${\Sigma_{ii}(X) \in [\epsilon, 1]}$ for all $i$ (see Appendix \ref{appendix:discussion-constraints}).
In practice, numerical stability often requires the diagonal entries of $\Sigma$ to be lower bounded by a small number (e.g. $10^{-6}$).
If $\mathcal{X}$ is compact (as for images) then such a $K$ is guaranteed to exist; if not, choosing $K$ very large yields an insignificant constraint.

Table~\ref{table:convergence-other} summarizes the rates of bias for some existing methods.
In contrast to our proposal, the assumptions of these estimators may in practice be difficult to verify.
For the estimator of \cite{krishnamurthy14icml}, both densities $p$ and $q$ must belong to the H\"older class of smoothness $s$, be supported on $[0,1]^d$ and satisfy $0<\eta_1 < p, q < \eta_2<\infty$ on the support for known constants $\eta_1, \eta_2$.
For that of \cite{nguyen10ratio}, the density ratio $p/q$ must satisfy $0<\eta_1 < p/q < \eta_2<\infty$ and belong to a function class $G$ whose \emph{bracketing entropy} (a measure of the complexity of a function class) is properly bounded. The condition on the bracketing entropy is quite strong and ensures that the density ratio is well behaved.
For the estimator of \cite{moon14ensemble}, both $p$ and $q$ must have the same bounded support and satisfy $0<\eta_1 < p, q < \eta_2<\infty$ on the support. $p$ and $q$ must have \emph{continuous bounded} derivatives of order $d$ (which is stronger than assumptions of [22]), and $f$ must have derivatives of order at least $d$.

In summary, the RAM estimator $D_f(\hat{Q}_Z^N \| P_Z)$ for $D_f(Q_Z \| P_Z)$ is \textbf{consistent} since it concentrates to its expectation $\E_{\XN}\bigl[D_f(\hat{Q}_Z^N \| P_Z)\bigr]$, which in turn converges to $D_f(Q_Z \| P_Z)$.
It is also \textbf{practical} because it can be efficiently estimated with Monte-Carlo sampling via RAM-MC.